\documentclass{article}
\usepackage[utf8]{inputenc}
\usepackage[T1]{fontenc}
\usepackage{float}
\usepackage{hyperref}
\usepackage{amsmath}
\usepackage{cleveref}
\usepackage{graphicx}
\usepackage{amsfonts} 
\title{MDMM Project}
\author{Matino Ruben \\ Ritacco Antonio }
\date{June 2020}

\begin{document}

\maketitle

\section{Problema (N. 4)}

A shipping company has to send a set of orders over a multi-modal transportation network (trains +
trucks), represented as a directed graph where each arc has a unit transportation cost and a
maximum capacity (total amount of goods that can travel along that arc). Train arcs have far lower
transportation cost, have stricter capacity (railways are few w.r.t. roads) and are “longer” (only
reach a subset of nodes), while truck arcs have a higher transportation cost but larger capacity and
connect all nodes. Each order is represented by a given amount of goods, a starting node and an
ending node. To avoid possible mis-services to users, all the goods pertaining to the same order
have to be shipped together, i.e., along a single path; furthermore, the length of the path must not
exceed a prescribed constant k (nor very small, but not very large either). The problem is to find a
minimum-cost set of paths with the required property that allow to satisfy all the orders while
satisfying the arc capacity constraints.


\section{Modello matematico}
Il modello matematico che ci è sembrato più adeguato è il Minimum Cost Multicommodity Flow adattandolo al nostro caso nel quale i flussi vanno distinti in base al tipo di instradamento ferroviario o stradale. Inoltre per soddisfare l'unicità dei cammini abbiamo aggiunto un vincolo \eqref{mod:v8} in cui imponiamo che dato un ordine, ogni nodo appartenente a quel cammino possa avere al più un arco uscente.
\subsection{Parametri}
\begin{itemize}
    \item \textbf{$N$} è l'insieme dei nodi.
    \item \textbf{$A_f$} è l'insieme degli archi ferroviari.
    \item \textbf{$A_s$} è l'insieme degli archi stradali.
    \item \textbf{$H$} è l'insieme delle degli ordini
    \item $u_{s_{ij}}$ capacità delle strada $(i,j) > 0$ ; $=0$ se $(i,j) \not\in A_s $ 
    \item $u_{f_{ij}}$ capacità delle ferrovia $(i,j) > 0$ ; $=0$ se $(i,j) \not\in A_f $ 
    \item $c_{s_{ij}}$ costo delle strada $(i,j)$
    \item $c_{f_{ij}}$ costo delle ferrovia $(i,j)$
\end{itemize}

\begin{itemize}
    \item $b_i^h = 0$ se $i$ è un nodo di trasporto per ordine $h$
    \item  $b_i^h <0$ se $i$ è un nodo origine per ordine $h$
    \item $b_i^h >0$ se $i$ è un nodo destinazione per ordine $h$
\end{itemize}

\subsection{Variabili Decisionali}
\begin{itemize}
    \item $f_{ij}^h$ $\in \mathbb{R}$  è la quantità di merce su ferrovia $(i,j)$ relativa all'ordine h 
    \item $s_{ij}^h$ $\in \mathbb{R}$ è la quantità di merce su strada $(i,j)$ relativa all'ordine h
    \item $y_{s_{ij}}^h$ $\in \{0,1\}$ arco di attivazione della strada $(i,j)$ relativa all'ordine h
    \item $y_{f_{ij}}^h$ $\in \{0,1\}$ arco di attivazione della ferrovia $(i,j)$ relativa all'ordine h
\end{itemize}

\subsection{Modello}

\label{problem1}
\begin{align}
    min (\sum_{(i,j) \in A_f}\sum_{h=1}^H c_{f_{ij}}^h \cdot f_{ij}^h) +
    (\sum_{(i,j) \in A_s}\sum_{h=1}^H c_{s_{ij}}^h \cdot s_{ij}^h)
    \label{mod:obj}
    \\
    (\sum_{(j)}f_{ji}^h + s_{ji}^h)- (\sum_{(j)}f_{ij}^h + s_{ij}^h) = b_i^h, \; \; \forall i \in N, h=1, ... H 
    \label{mod:v2}
    \\
    f_{ij}^h \leq u_{f_{ij}} \cdot y_{f_{ij}}^h, \; \; \forall (i,j) \in A_f
    \label{mod:v3}
    \\
    s_{ij}^h \leq u_{s_{ij}} \cdot y_{s_{ij}}^h, \; \; \forall (i,j) \in A_s
    \label{mod:v4}
    \\
    \sum_{(h=1)}^H f_{ij}^h \leq u_{f_{ij}}, \; \; \forall (i,j) \in A_f
    \label{mod:v5}
    \\
    \sum_{(h=1)}^H s_{ij}^h \leq u_{s_{ij}}, \; \; \forall (i,j) \in A_s
    \label{mod:v6}
    \\
    \sum_{(i,j)} y_{f_{ij}}^h + y_{s_{ij}}^h \leq K, \; \; h=1, ... H
    \label{mod:v7}
    \\
    \sum_{j} y_{f_{ij}}^h + y_{s_{ij}}^h \leq 1, \; \; \forall i \in \textit{N} , h=1, ... H
    \label{mod:v8}
    \\
    y_{f_{ij}}^h \in \{0,1\}, \; \; \forall (i,j) \in A_f, h=1, ... H
    \label{mod:v9}
    \\
    y_{s_{ij}}^h \in \{0,1\}, \; \; \forall (i,j) \in A_s, h=1, ... H
    \label{mod:v10}
\end{align}

\begin{itemize}
    \item In \eqref{mod:obj} è presentata la funzione obiettivo. Questa consiste nella minimizzazione, per ogni ordine $h$, del costo di instradare la merce lungo una tratta composta possibilmente da ferrovie e/o strade.
    \item In \eqref{mod:v2} è presentato il vincolo di bilancio di flusso per ogni ordine $h \in H$.
    \item Le disequazioni \eqref{mod:v3} ed \eqref{mod:v4} rappresentano i vincoli di attivazione degli archi.
    \item Le disequazioni \eqref{mod:v5} ed \eqref{mod:v6} rappresentano i vincoli sulla capacita' degli archi.
    \item La disequazione \eqref{mod:v7} modella il vincolo di lunghezza massima del cammino $\leq K$.
    \item Infine, la disequazione \eqref{mod:v8} modella il vincolo di unicita' del cammino per ogni commodity.
    \item In \eqref{mod:v9} e \eqref{mod:v10} sono invece presenti i vincoli di interezza per le attivazioni degli archi.

\end{itemize}

\end{document}



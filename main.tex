\documentclass{article}
\usepackage[utf8]{inputenc}
\usepackage[T1]{fontenc}
\usepackage{float}
\usepackage{hyperref}
\usepackage{amsmath}
\usepackage{cleveref}
\usepackage{graphicx}
\usepackage{amsfonts} 
\title{MDMM Project}
\author{Matino Ruben \\ Ritacco Antonio }
\date{June 2020}

\begin{document}

\maketitle

\section{Problema (N. 4)}

A shipping company has to send a set of orders over a multi-modal transportation network (trains +
trucks), represented as a directed graph where each arc has a unit transportation cost and a
maximum capacity (total amount of goods that can travel along that arc). Train arcs have far lower
transportation cost, have stricter capacity (railways are few w.r.t. roads) and are “longer” (only
reach a subset of nodes), while truck arcs have a higher transportation cost but larger capacity and
connect all nodes. Each order is represented by a given amount of goods, a starting node and an
ending node. To avoid possible mis-services to users, all the goods pertaining to the same order
have to be shipped together, i.e., along a single path; furthermore, the length of the path must not
exceed a prescribed constant k (nor very small, but not very large either). The problem is to find a
minimum-cost set of paths with the required property that allow to satisfy all the orders while
satisfying the arc capacity constraints.


\section{Modello matematico}
Il modello matematico che ci è sembrato più adeguato è il Minimum Cost Multicommodity Flow adattandolo al nostro caso nel quale i flussi vanno distinti in base al tipo di instradamento ferroviario o stradale e la merce viaggia su un unico cammino. Questo modello non prevede la possibilit\'a di cambiare mezzo di trasporto in un dato nodo per un dato ordine (per ottenere questa possibilit\'a le variabili $s_h$ ed $f_h$ andrebbero estese ad ogni nodo). Per soddisfare l'unicit\'a dei cammini abbiamo aggiunto un vincolo \eqref{mod:v6} in cui imponiamo che dato un ordine, ogni nodo appartenente a quel cammino possa avere al più un arco uscente. Per soddisfare la lunghezza massima del cammino abbiamo aggiunto il vincolo \eqref{mod:v5}.

\subsection{Parametri}
\begin{itemize}
    \item \textbf{$N$} è l'insieme dei nodi.
    \item \textbf{$A_f$} è l'insieme degli archi ferroviari.
    \item \textbf{$A_s$} è l'insieme degli archi stradali.
    \item \textbf{$H$} è l'insieme delle degli ordini
    \item $u_{s_{ij}}$ capacità delle strada $(i,j) > 0$ ; $=0$ se $(i,j) \not\in A_s $ 
    \item $u_{f_{ij}}$ capacità delle ferrovia $(i,j) > 0$ ; $=0$ se $(i,j) \not\in A_f $ 
    \item $c_{s_{ij}}$ costo delle strada $(i,j)$
    \item $c_{f_{ij}}$ costo delle ferrovia $(i,j)$
    \item $d_h$ quantita' merce $h$
\end{itemize}

\begin{itemize}
    \item $b_i^h = 0$ se $i$ è un nodo di trasporto per ordine $h$
    \item  $b_i^h =-1$ se $i$ è un nodo origine per ordine $h$
    \item $b_i^h =1$ se $i$ è un nodo destinazione per ordine $h$
\end{itemize}

\subsection{Variabili Decisionali}
\begin{itemize}
    \item $y_{ij}^h$ $\in \{0,1\}$ variabile attivazione dell'arco $(i,j)$ relativa all'ordine h
    \item $f^h$ $\in \{0,1\}$, =1 se la merce h va in ferrovia, 0 ow
    \item $s^h$ $\in \{0,1\}$, =1 se la merce h va in strada, 0 ow
\end{itemize}

\subsection{Modello}

\label{problem1}
\begin{align}
    min (\sum_{(i,j) \in A_f}\sum_{h=1}^H c_{f_{ij}}^h \cdot f^h \cdot d^h) +
    (\sum_{(i,j) \in A_s}\sum_{h=1}^H c_{s_{ij}}^h \cdot s^h \cdot d^h )
    \label{mod:obj}
    \\
    (\sum_{j}y_{ji}^h)- (\sum_{j}y_{ij}^h ) = b_i^h, \; \; \forall i \in N, h=1, ... H
    \label{mod:v2}
    \\
    \sum_{(h=1)}^H y_{ij}^h \cdot f_h \cdot d_h \leq u_{f_{ij}}, \; \; \forall (i,j) \in A_f
    \label{mod:v3}
    \\
    \sum_{(h=1)}^H y_{ij}^h \cdot s_h \cdot d_h \leq u_{s_{ij}}, \; \; \forall (i,j) \in A_s
    \label{mod:v4}
    \\
    \sum_{(i,j)} y_{ij}^h  \leq K, \; \; h=1, ... H
    \label{mod:v5}
    \\
    \sum_{j} y_{ij}^h \leq 1, \; \; \forall i \in \textit{N} , h=1, ... H
    \label{mod:v6}
    \\
    s^h + f^h = 1 , \; \;  h=1, ... H
    \label{mod:v7}
    \\
    y_{ij}^h, s^h, f^h \in \{0,1\}, \; \; \forall (i,j) \in A_s \cup A_f , h=1, ... H
    \label{mod:v8}
\end{align}

\begin{itemize}
    \item In \eqref{mod:obj} è presentata la funzione obiettivo. Questa consiste nella minimizzazione, per ogni ordine $h$, del costo di instradare la merce lungo una tratta composta possibilmente da ferrovie e/o strade.
    \item In \eqref{mod:v2} serve a definire i nodi destinazione, origine ed intermedi per ogni ordine $h \in H$.
    \item Le disequazioni \eqref{mod:v3} ed \eqref{mod:v4} rappresentano i vincoli di capacit\'a degli archi.
    \item Le disequazioni \eqref{mod:v5} modella il vincolo di lunghezza massima del cammino $\leq K$. 
    \item La \eqref{mod:v6} modella il vincolo di unicita' del cammino per ogni commodity.
    \item Infine, la disequazione \eqref{mod:v7} garantisce che per ogni ordine $h$ venga trasportato o su strada o su ferrovia (ma non su entrambi).
    \item In \eqref{mod:v8} sono invece presenti i vincoli di interezza per le variabili di attivazione degli archi e per le variabili di scelta del mezzo di trasporto.

\end{itemize}

\end{document}


